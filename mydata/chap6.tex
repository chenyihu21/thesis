% !TeX root = ../cyh.tex

\chapter{结论与展望}

\section{结论}

到目前为止,我们已经基本完成了 starry-next 内存管理模块及其接口的设计与实现,并且对其进行了测试。我们的实现与 Linux 内核的内存管理模块接口基本一致,并且在功能测试中表现良好,
性能方面则需要进一步测试和优化。
具体的代码已开源至 \href{https://github.com/chenyihu21/starry-next}{GitHub}。

在开发过程中,由于操作系统的各模块间存在一定的依赖关系,
我们使用了多种协同开发 OS 的方法,包括使用 GitHub 的 issue 和 pull request 进行协作,
根据模块和系统调用分配任务,以通过某个测例为目的进行调试等。
我们不仅参与了宏内核接口的设计、实现与测试,还一定程度上参与了基座微内核 ArceOS 的修改。
另外,我们还将开发过程中的问题和解决方法记录在了实验报告中,以便后续的开发工作中可以参考。
希望这些工作能为后续组件化操作系统的开发提供一些参考和帮助。

\section{展望}

在未来的工作中,我们计划进一步完善 starry-next 内存管理模块及其接口,
包括完善当前仅实现部分功能的系统调用接口、优化内存管理模块的性能等方面。另外,当前 starry-next 内存管理模块还未支持各类页面置换算法,
我们计划在未来的工作中增加对不同页面置换算法的支持,以提高内存管理模块的性能和效率。
同时,增加对内存管理模块的安全性和稳定性的支持也会纳入未来的计划,
以确保内存管理模块在实际应用中能够稳定运行。