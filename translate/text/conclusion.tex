% \section{Conclusions}
% In this paper we introduced \OSname{}, the first embedded Rust operating system for microcontrollers supporting both single- and multicore preemptive scheduling combined with asynchronous Rust.
% We assessed experimentally how 
% a unique multicore scheduler can be the convenient default on all supported multicore platforms. 
% Still, applications 

% can opt out of multicore scheduling when parallelization is unneeded.

% \OSname{} thus enriches the set of available open source tools 
% for secure and efficient distributed computing applications involving sensors/actuators or small networked devices using 32-bit MCUs such as ARM Cortex\nobreakdash-M, RISC\nobreakdash-V, or ESP32.

\section{结论}
在本文中,我们介绍了\OSname{},这是第一个支持单核和多核抢占式调度以及异步Rust的微控制器嵌入式Rust操作系统。我们通过实验评估了在所有支持的多核平台上,独特的多核调度器如何可以作为便捷的默认选项。然而,当并行化并非必要时,应用程序可以选择退出多核调度。
因此,\OSname{}丰富了现有的开源工具集,适用于涉及传感器/执行器或使用32位微控制器(如ARM Cortex\nobreakdash-M、RISC\nobreakdash-V或ESP32)的小型网络设备的安全高效分布式计算应用。

%Researchers can use \OSname{} as common playground for experimental implementation and performance studies involving embedded Rust software on all the relevant 32-bit microcontroller architectures (ARM Cortex\nobreakdash-M, RISC\nobreakdash-V, ESP32).
%Industry users from diverse verticals can use \OSname{} to accelerate their transition from C/C++ towards safer embedded Rust, while benefiting from a high degree of hardware agility. %, ingrained by design. 
%\textbf{Code Availability ---} \OSname{} is fully open source~\cite{ariel-os-repo}.% and  maintained on a public git repository by 10+ people located in 4 countries across the European Union.

%in terms of reusing their software (business logic) on a wide variety of microcontroller architectures and commercially available off-the-shelf boards in this category.

